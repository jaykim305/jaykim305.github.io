\documentclass[margin,line,letter]{cv}
\usepackage{url}
\topmargin -.3in
% \bmargin -.3in
\oddsidemargin -.5in
\evensidemargin -.5in
\textwidth=6in
\textheight=10in
\itemsep=0in
\parsep=0in
\newsectionwidth{1in}

\def \pubitem {\vspace{0.07in}\item}

\newenvironment{list1}{
  \begin{list}{\ding{113}}{%
      \setlength{\itemsep}{0in}
      \setlength{\parsep}{0in} \setlength{\parskip}{0in}
      \setlength{\topsep}{0in} \setlength{\partopsep}{0in}
      \setlength{\leftmargin}{0.17in}}}{\end{list}}
\newenvironment{list2}{
  \begin{list}{$\bullet$}{%
      \setlength{\itemsep}{0in}
      \setlength{\parsep}{0in} \setlength{\parskip}{0in}
      \setlength{\topsep}{0in} \setlength{\partopsep}{0in}
      \setlength{\leftmargin}{0.2in}}}{\end{list}}

\usepackage{times}
\usepackage{cite}
\usepackage{hyperref}
\usepackage{graphicx}
\usepackage{enumitem}

\newcommand{\video}[1]{\underline{\textsf{\small{#1}}}}

\DeclareGraphicsExtensions{.eps,.jpg,.png,.gif}
\DeclareGraphicsRule{.png}{eps}{.bb}{}

\begin{document}
\name{\huge Jaehong Kim  \vspace*{.1in}}

\begin{resume}

\section{\sc Contact \\}
\begin{tabular}{@{} p{2.94in} r}
Ph.D Candidate\\
School of Electrical Engineering, KAIST \\
{\it Phone:} (+82)10-4105-7379 & Kim Byung Ho IT Building (N1) \#817 \\
{\it Email:} jaehong950305@gmail.com & KAIST, 291 Daehak-ro, Yuseong-gu, Daejeon \\
{\it Webpage:} \urlstyle{sf}\url{https://jaykim305.github.io/} & 305-701, Republic of Korea \\
\end{tabular}

\section{\sc Research Interests \\}
High Performance Networked Systems, Deep Learning based Video Delivery, Video Analytics  \\

\section{\sc Education \\}

\textbf{Korea Advanced Institute of Science and Technology (KAIST)} \hfill \textsc{Feb.} 2020 $\sim$ Present \\
Ph.D., in School of Electrical Engineering (Advisor: Prof. Dongsu Han) \vspace{0.2cm} \\ 
\textbf{Korea Advanced Institute of Science and Technology (KAIST)} \hfill \textsc{Aug.} 2018 $\sim$ \textsc{Feb.} 2020 \\
M.S., in School of Electrical Engineering (Advisor: Prof. Dongsu Han) \vspace{0.2cm} \\ 
\textbf{Korea Advanced Institute of Science and Technology (KAIST)} \hfill \textsc{Feb.} 2014 $\sim$ \textsc{Aug.} 2018 \\
B.S., in School of Electrical Engineering  (Cum Laude) \par 

% \vspace{0.5cm}

\section{\sc Publications \\}
\textbf{Conference}
(* denotes equal contribution.)
\begin{itemize}[leftmargin=.2in]
	\pubitem {\textbf{Co-optimizing for Flow Completion Time in Radio Access Network} \\ \underline{Jaehong Kim}, Yunheon Lee, Hwijoon Lim, Youngmok Jung, Song Min Kim, and Dongsu Han \\ \textbf{ACM CoNEXT 2022} (Acceptance Rate 29/151: 19.2\%)}	
	\pubitem {\textbf{NeuroScaler: Neural Video Enhancement at Scale} \\ Hyunho Yeo, Hwijoon Lim, \underline{Jaehong Kim}, Yongmok Jung, Juncheol Ye and Dongsu Han \\ \textbf{ACM SIGCOMM 2022} (Acceptance Rate 55/279: 19.7\%)}	
	\pubitem {\textbf{Neural-Enhanced Live Streaming: Improving Live Video Ingest via Online Learning} \\ \underline{Jaehong Kim*}, Youngmok Jung*, Hyunho Yeo, Juncheol Ye and Dongsu Han \\ \textbf{ACM SIGCOMM 2020} (Acceptance Rate 53/250: 21.2\%)}
	\pubitem {\textbf{Neural Adaptive Content-aware Internet Video Delivery} \\ Hyunho Yeo, Youngmok Jung, \underline{Jaehong Kim}, Jinwoo Shin and Dongsu Han \\ \textbf{USENIX OSDI 2018} (Acceptance Rate 47/257: 18.2\%)}
\end{itemize}

% \vspace{0.3cm}

%\textbf{Journal}
%\begin{enumerate}
%	\pubitem {}
%\end{enumerate}
\vspace{0.2cm}

%\textbf{Poster}
%\begin{enumerate}
%	\pubitem {}
%\end{enumerate}


%\newpage

\section{\sc Honors and\\ Awards \\}
\begin{itemize}[leftmargin=*]
	\item \textbf{28th Samsung Humantech Paper Award} \hfill Samsung Electronics, \textsc{Feb.} 2022 \\ Gold Prize (Co-author), Communication \& Networks. \par 
	\item \textbf{\href{https://breakthroughs.kaist.ac.kr/?post_no=1913}{KAIST Breakthroughs of the Year 2021 Spring}} \hfill KAIST, 2021 \par 
	\item \textbf{Donghwa Industry Moon Daewon AI Research Scholarship} \hfill KAIST, 2020 \par 
	\item \textbf{USENIX OSDI Student Grant} \hfill USENIX, 2018 \par 
\end{itemize}

\vspace{0.3cm}

\section{\sc Research \\ Projects \\}   
\begin{itemize}[leftmargin=*]
	\item{\textbf{Direct Volume Render Streaming} \hfill \textsc{Apr.} 2022 $\sim$ \textsc{July}. 2022 
	\\ Implemented a DICOM 3D visualization app prototype for Oculus Quest2 using Nvidia \textbf{CloudXR} and Unity. 
	Funded by INUCreative Inc.
	\href{https://youtu.be/H_05eXnPR8I}{\video{Demo video link (CloudXR)}}, \href{https://youtu.be/D6MKOwsuRXU}{\video{Demo video link (Unity)}}}
	\pubitem{\textbf{Neural Video Enhancement at Scale} \hfill \textsc{Oct.} 2021 $\sim$ \textsc{Dec.} 2021}
	\pubitem{\textbf{Optimizing downlink scheduling in Radio Access Networks} \hfill \textsc{Aug.} 2020 $\sim$ Present
	\\ Designed a practical flow scheduler for LTE/5G xNodeBs that achieves low-latency for Interactive traffic. 
	\\ Implemented the system on top of \textbf{srsRAN} (i.e., open-source LTE/5G software radio suite) 
	and \textbf{NS-3}. The scheduler can reduce webpage load time of Android phones up to \textbf{34\%}.
	Funded by \textbf{Samsung Electronics Co., Ltd. Modem S/W R\&D Group.}}
	\pubitem{\textbf{Deploying Credit-based Proactive Transport for Datacenter Networks}} \hfill \textsc{July.} 2020 $\sim$ \textsc{Jan.} 2021
	\pubitem{\textbf{Neural-enhanced Live Streaming (LiveNAS)} \hfill \textsc{Nov.} 2018 $\sim$ \textsc{July.} 2020
	\\ Designed a new live ingest system that enhances the origin live stream’s quality with online-trained super-resolution DNNs at the ingest server. The system delivers up to \textbf{69\%} QoE improvement. 
	Implemented client, server with \textbf{WebRTC}, \textbf{PyTorch} and \textbf{ffmpeg}. Led the project as a \textbf{team leader}.}
	\pubitem{\textbf{Neural-enhanced Adaptive Streaming (NAS)} \hfill \textsc{Nov.} 2017 $\sim$ \textsc{Oct.} 2018 \\ Designed a new video delivery system that integrates super-resolution DNNs with adaptive streaming. 
	Implemented \textbf{dash.js} that handles DNN integrated ABR and super-resolution on MPEG video chunks.}
\end{itemize}

\vspace{0.2cm}

\section{\sc Invited \\ Talks}
\begin{itemize}[leftmargin=*]
	\item{\textbf{Neural-Enhanced Live Streaming: Improving Live Video Ingest via Online Learning} \\ Conference talk at SIGCOMM, Aug., 2020. 
	\href{https://youtu.be/1giVlO6Rumg}{\video{10 min talk video link}}, \href{https://youtu.be/avkSHrXlBSA}{\video{20 min talk video link}}}
	\item{\textbf{Neural Adaptive Content-aware Internet Video Delivery} \\ Poster \& Demo Session at OSDI, Oct., 2018. 
	\href{https://youtu.be/THGMsqFOxWU}{\video{Demo video link}}}
	% \item{\textbf{NNStreamer Conference 2022 (NC22-Seoul)} 
	% \\ Research talk at \href{https://github.com/nnstreamer/nnstreamer}{NNStreamer} Workshop, Feb., 2022}
\end{itemize}

%\section{\sc Software\\ Published \\}

%\vspace{0.5cm}

%\section{\sc Reviewer\\ Experience\\}

%\vspace{0.5cm}


\section{\sc Teaching Experience}
\textbf{Teaching Assistant}
\begin{itemize}[leftmargin=.2in] 
	\item{\textbf{Advanced Computer Networking and Cloud Computing (EE618)}
	\hfill \textsc{Spring} 2021}
	\item{\textbf{Network Programming (EE324)}
	\hfill \textsc{Fall} 2020, \textsc{Fall} 2021}
	\item{\textbf{SK Hynix ASK Program}
	\hfill \textsc{Aug.} 2020}
	\item{\textbf{Systems and Applications of Artificial Intelligence and Machine Learning (EE793)}
	\hfill \textsc{Spring} 2020}
	\item{\textbf{Programming Structures for Electrical Engineering (EE209)}
	\hfill \textsc{Spring}\&\textsc{Fall} 2019, \href{https://ee209.kaist.ac.kr/}{\textsc{Spring}\&\textsc{Fall} 2022}}
\end{itemize}
\vspace{0.1cm}
% \section{\sc Courses}
% \emph{Systems and Applications of Artificial Intelligence and Machine Learning (EE793)} \hfill \textsc{Spring} 2020 \\
% \emph{Computer Architecture (EE511)} \hfill \textsc{Fall} 2019 \\
% \emph{Startup Finance (KEI530)} \hfill \textsc{Fall} 2019 \\
% \emph{GPU Proprogramming and Its Application (EE817)} \hfill \textsc{Spring} 2019 \\
% \emph{Advanced Computer Networking and Cloud Computing (EE618)} \hfill \textsc{Spring} 2019 \\
% \emph{Advanced Image Restoration and Quality Enhancement (EE838)} \hfill \textsc{Fall} 2018 \\
% \emph{Recent Advances in Deep Learning (EE807)} \hfill \textsc{Fall} 2018 \\
% \emph{Operating Systems and System Programming for Electrical Engineering (EE415)} \hfill \textsc{Spring} 2018 \\
% \emph{Deep Learning and AlphaGo (EE488)} \hfill \textsc{Fall} 2017 \\
%\vspace{0.3cm}
\section{\sc Proficient Skills \\}
Programming Languages: C, C++, Python, UNIX shell scripting, Latex, JavaScript \\
Tools \& Frameworks: dash.js, ffmpeg, NS-3 Simulator, srsRAN, Docker, Azure Kinect \\ 
Deep Learning Frameworks: Tensorflow, PyTorch \\
Languages: Korean (native), English (IBT TOEFL 106, test date: 2015.08.22)

\end{resume}
\end{document}
